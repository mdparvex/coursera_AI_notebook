
% Default to the notebook output style

    


% Inherit from the specified cell style.




    
\documentclass[11pt]{article}

    
    
    \usepackage[T1]{fontenc}
    % Nicer default font (+ math font) than Computer Modern for most use cases
    \usepackage{mathpazo}

    % Basic figure setup, for now with no caption control since it's done
    % automatically by Pandoc (which extracts ![](path) syntax from Markdown).
    \usepackage{graphicx}
    % We will generate all images so they have a width \maxwidth. This means
    % that they will get their normal width if they fit onto the page, but
    % are scaled down if they would overflow the margins.
    \makeatletter
    \def\maxwidth{\ifdim\Gin@nat@width>\linewidth\linewidth
    \else\Gin@nat@width\fi}
    \makeatother
    \let\Oldincludegraphics\includegraphics
    % Set max figure width to be 80% of text width, for now hardcoded.
    \renewcommand{\includegraphics}[1]{\Oldincludegraphics[width=.8\maxwidth]{#1}}
    % Ensure that by default, figures have no caption (until we provide a
    % proper Figure object with a Caption API and a way to capture that
    % in the conversion process - todo).
    \usepackage{caption}
    \DeclareCaptionLabelFormat{nolabel}{}
    \captionsetup{labelformat=nolabel}

    \usepackage{adjustbox} % Used to constrain images to a maximum size 
    \usepackage{xcolor} % Allow colors to be defined
    \usepackage{enumerate} % Needed for markdown enumerations to work
    \usepackage{geometry} % Used to adjust the document margins
    \usepackage{amsmath} % Equations
    \usepackage{amssymb} % Equations
    \usepackage{textcomp} % defines textquotesingle
    % Hack from http://tex.stackexchange.com/a/47451/13684:
    \AtBeginDocument{%
        \def\PYZsq{\textquotesingle}% Upright quotes in Pygmentized code
    }
    \usepackage{upquote} % Upright quotes for verbatim code
    \usepackage{eurosym} % defines \euro
    \usepackage[mathletters]{ucs} % Extended unicode (utf-8) support
    \usepackage[utf8x]{inputenc} % Allow utf-8 characters in the tex document
    \usepackage{fancyvrb} % verbatim replacement that allows latex
    \usepackage{grffile} % extends the file name processing of package graphics 
                         % to support a larger range 
    % The hyperref package gives us a pdf with properly built
    % internal navigation ('pdf bookmarks' for the table of contents,
    % internal cross-reference links, web links for URLs, etc.)
    \usepackage{hyperref}
    \usepackage{longtable} % longtable support required by pandoc >1.10
    \usepackage{booktabs}  % table support for pandoc > 1.12.2
    \usepackage[inline]{enumitem} % IRkernel/repr support (it uses the enumerate* environment)
    \usepackage[normalem]{ulem} % ulem is needed to support strikethroughs (\sout)
                                % normalem makes italics be italics, not underlines
    

    
    
    % Colors for the hyperref package
    \definecolor{urlcolor}{rgb}{0,.145,.698}
    \definecolor{linkcolor}{rgb}{.71,0.21,0.01}
    \definecolor{citecolor}{rgb}{.12,.54,.11}

    % ANSI colors
    \definecolor{ansi-black}{HTML}{3E424D}
    \definecolor{ansi-black-intense}{HTML}{282C36}
    \definecolor{ansi-red}{HTML}{E75C58}
    \definecolor{ansi-red-intense}{HTML}{B22B31}
    \definecolor{ansi-green}{HTML}{00A250}
    \definecolor{ansi-green-intense}{HTML}{007427}
    \definecolor{ansi-yellow}{HTML}{DDB62B}
    \definecolor{ansi-yellow-intense}{HTML}{B27D12}
    \definecolor{ansi-blue}{HTML}{208FFB}
    \definecolor{ansi-blue-intense}{HTML}{0065CA}
    \definecolor{ansi-magenta}{HTML}{D160C4}
    \definecolor{ansi-magenta-intense}{HTML}{A03196}
    \definecolor{ansi-cyan}{HTML}{60C6C8}
    \definecolor{ansi-cyan-intense}{HTML}{258F8F}
    \definecolor{ansi-white}{HTML}{C5C1B4}
    \definecolor{ansi-white-intense}{HTML}{A1A6B2}

    % commands and environments needed by pandoc snippets
    % extracted from the output of `pandoc -s`
    \providecommand{\tightlist}{%
      \setlength{\itemsep}{0pt}\setlength{\parskip}{0pt}}
    \DefineVerbatimEnvironment{Highlighting}{Verbatim}{commandchars=\\\{\}}
    % Add ',fontsize=\small' for more characters per line
    \newenvironment{Shaded}{}{}
    \newcommand{\KeywordTok}[1]{\textcolor[rgb]{0.00,0.44,0.13}{\textbf{{#1}}}}
    \newcommand{\DataTypeTok}[1]{\textcolor[rgb]{0.56,0.13,0.00}{{#1}}}
    \newcommand{\DecValTok}[1]{\textcolor[rgb]{0.25,0.63,0.44}{{#1}}}
    \newcommand{\BaseNTok}[1]{\textcolor[rgb]{0.25,0.63,0.44}{{#1}}}
    \newcommand{\FloatTok}[1]{\textcolor[rgb]{0.25,0.63,0.44}{{#1}}}
    \newcommand{\CharTok}[1]{\textcolor[rgb]{0.25,0.44,0.63}{{#1}}}
    \newcommand{\StringTok}[1]{\textcolor[rgb]{0.25,0.44,0.63}{{#1}}}
    \newcommand{\CommentTok}[1]{\textcolor[rgb]{0.38,0.63,0.69}{\textit{{#1}}}}
    \newcommand{\OtherTok}[1]{\textcolor[rgb]{0.00,0.44,0.13}{{#1}}}
    \newcommand{\AlertTok}[1]{\textcolor[rgb]{1.00,0.00,0.00}{\textbf{{#1}}}}
    \newcommand{\FunctionTok}[1]{\textcolor[rgb]{0.02,0.16,0.49}{{#1}}}
    \newcommand{\RegionMarkerTok}[1]{{#1}}
    \newcommand{\ErrorTok}[1]{\textcolor[rgb]{1.00,0.00,0.00}{\textbf{{#1}}}}
    \newcommand{\NormalTok}[1]{{#1}}
    
    % Additional commands for more recent versions of Pandoc
    \newcommand{\ConstantTok}[1]{\textcolor[rgb]{0.53,0.00,0.00}{{#1}}}
    \newcommand{\SpecialCharTok}[1]{\textcolor[rgb]{0.25,0.44,0.63}{{#1}}}
    \newcommand{\VerbatimStringTok}[1]{\textcolor[rgb]{0.25,0.44,0.63}{{#1}}}
    \newcommand{\SpecialStringTok}[1]{\textcolor[rgb]{0.73,0.40,0.53}{{#1}}}
    \newcommand{\ImportTok}[1]{{#1}}
    \newcommand{\DocumentationTok}[1]{\textcolor[rgb]{0.73,0.13,0.13}{\textit{{#1}}}}
    \newcommand{\AnnotationTok}[1]{\textcolor[rgb]{0.38,0.63,0.69}{\textbf{\textit{{#1}}}}}
    \newcommand{\CommentVarTok}[1]{\textcolor[rgb]{0.38,0.63,0.69}{\textbf{\textit{{#1}}}}}
    \newcommand{\VariableTok}[1]{\textcolor[rgb]{0.10,0.09,0.49}{{#1}}}
    \newcommand{\ControlFlowTok}[1]{\textcolor[rgb]{0.00,0.44,0.13}{\textbf{{#1}}}}
    \newcommand{\OperatorTok}[1]{\textcolor[rgb]{0.40,0.40,0.40}{{#1}}}
    \newcommand{\BuiltInTok}[1]{{#1}}
    \newcommand{\ExtensionTok}[1]{{#1}}
    \newcommand{\PreprocessorTok}[1]{\textcolor[rgb]{0.74,0.48,0.00}{{#1}}}
    \newcommand{\AttributeTok}[1]{\textcolor[rgb]{0.49,0.56,0.16}{{#1}}}
    \newcommand{\InformationTok}[1]{\textcolor[rgb]{0.38,0.63,0.69}{\textbf{\textit{{#1}}}}}
    \newcommand{\WarningTok}[1]{\textcolor[rgb]{0.38,0.63,0.69}{\textbf{\textit{{#1}}}}}
    
    
    % Define a nice break command that doesn't care if a line doesn't already
    % exist.
    \def\br{\hspace*{\fill} \\* }
    % Math Jax compatability definitions
    \def\gt{>}
    \def\lt{<}
    % Document parameters
    \title{AI4M\_C1\_W1\_lecture\_ex\_01}
    
    
    

    % Pygments definitions
    
\makeatletter
\def\PY@reset{\let\PY@it=\relax \let\PY@bf=\relax%
    \let\PY@ul=\relax \let\PY@tc=\relax%
    \let\PY@bc=\relax \let\PY@ff=\relax}
\def\PY@tok#1{\csname PY@tok@#1\endcsname}
\def\PY@toks#1+{\ifx\relax#1\empty\else%
    \PY@tok{#1}\expandafter\PY@toks\fi}
\def\PY@do#1{\PY@bc{\PY@tc{\PY@ul{%
    \PY@it{\PY@bf{\PY@ff{#1}}}}}}}
\def\PY#1#2{\PY@reset\PY@toks#1+\relax+\PY@do{#2}}

\expandafter\def\csname PY@tok@w\endcsname{\def\PY@tc##1{\textcolor[rgb]{0.73,0.73,0.73}{##1}}}
\expandafter\def\csname PY@tok@c\endcsname{\let\PY@it=\textit\def\PY@tc##1{\textcolor[rgb]{0.25,0.50,0.50}{##1}}}
\expandafter\def\csname PY@tok@cp\endcsname{\def\PY@tc##1{\textcolor[rgb]{0.74,0.48,0.00}{##1}}}
\expandafter\def\csname PY@tok@k\endcsname{\let\PY@bf=\textbf\def\PY@tc##1{\textcolor[rgb]{0.00,0.50,0.00}{##1}}}
\expandafter\def\csname PY@tok@kp\endcsname{\def\PY@tc##1{\textcolor[rgb]{0.00,0.50,0.00}{##1}}}
\expandafter\def\csname PY@tok@kt\endcsname{\def\PY@tc##1{\textcolor[rgb]{0.69,0.00,0.25}{##1}}}
\expandafter\def\csname PY@tok@o\endcsname{\def\PY@tc##1{\textcolor[rgb]{0.40,0.40,0.40}{##1}}}
\expandafter\def\csname PY@tok@ow\endcsname{\let\PY@bf=\textbf\def\PY@tc##1{\textcolor[rgb]{0.67,0.13,1.00}{##1}}}
\expandafter\def\csname PY@tok@nb\endcsname{\def\PY@tc##1{\textcolor[rgb]{0.00,0.50,0.00}{##1}}}
\expandafter\def\csname PY@tok@nf\endcsname{\def\PY@tc##1{\textcolor[rgb]{0.00,0.00,1.00}{##1}}}
\expandafter\def\csname PY@tok@nc\endcsname{\let\PY@bf=\textbf\def\PY@tc##1{\textcolor[rgb]{0.00,0.00,1.00}{##1}}}
\expandafter\def\csname PY@tok@nn\endcsname{\let\PY@bf=\textbf\def\PY@tc##1{\textcolor[rgb]{0.00,0.00,1.00}{##1}}}
\expandafter\def\csname PY@tok@ne\endcsname{\let\PY@bf=\textbf\def\PY@tc##1{\textcolor[rgb]{0.82,0.25,0.23}{##1}}}
\expandafter\def\csname PY@tok@nv\endcsname{\def\PY@tc##1{\textcolor[rgb]{0.10,0.09,0.49}{##1}}}
\expandafter\def\csname PY@tok@no\endcsname{\def\PY@tc##1{\textcolor[rgb]{0.53,0.00,0.00}{##1}}}
\expandafter\def\csname PY@tok@nl\endcsname{\def\PY@tc##1{\textcolor[rgb]{0.63,0.63,0.00}{##1}}}
\expandafter\def\csname PY@tok@ni\endcsname{\let\PY@bf=\textbf\def\PY@tc##1{\textcolor[rgb]{0.60,0.60,0.60}{##1}}}
\expandafter\def\csname PY@tok@na\endcsname{\def\PY@tc##1{\textcolor[rgb]{0.49,0.56,0.16}{##1}}}
\expandafter\def\csname PY@tok@nt\endcsname{\let\PY@bf=\textbf\def\PY@tc##1{\textcolor[rgb]{0.00,0.50,0.00}{##1}}}
\expandafter\def\csname PY@tok@nd\endcsname{\def\PY@tc##1{\textcolor[rgb]{0.67,0.13,1.00}{##1}}}
\expandafter\def\csname PY@tok@s\endcsname{\def\PY@tc##1{\textcolor[rgb]{0.73,0.13,0.13}{##1}}}
\expandafter\def\csname PY@tok@sd\endcsname{\let\PY@it=\textit\def\PY@tc##1{\textcolor[rgb]{0.73,0.13,0.13}{##1}}}
\expandafter\def\csname PY@tok@si\endcsname{\let\PY@bf=\textbf\def\PY@tc##1{\textcolor[rgb]{0.73,0.40,0.53}{##1}}}
\expandafter\def\csname PY@tok@se\endcsname{\let\PY@bf=\textbf\def\PY@tc##1{\textcolor[rgb]{0.73,0.40,0.13}{##1}}}
\expandafter\def\csname PY@tok@sr\endcsname{\def\PY@tc##1{\textcolor[rgb]{0.73,0.40,0.53}{##1}}}
\expandafter\def\csname PY@tok@ss\endcsname{\def\PY@tc##1{\textcolor[rgb]{0.10,0.09,0.49}{##1}}}
\expandafter\def\csname PY@tok@sx\endcsname{\def\PY@tc##1{\textcolor[rgb]{0.00,0.50,0.00}{##1}}}
\expandafter\def\csname PY@tok@m\endcsname{\def\PY@tc##1{\textcolor[rgb]{0.40,0.40,0.40}{##1}}}
\expandafter\def\csname PY@tok@gh\endcsname{\let\PY@bf=\textbf\def\PY@tc##1{\textcolor[rgb]{0.00,0.00,0.50}{##1}}}
\expandafter\def\csname PY@tok@gu\endcsname{\let\PY@bf=\textbf\def\PY@tc##1{\textcolor[rgb]{0.50,0.00,0.50}{##1}}}
\expandafter\def\csname PY@tok@gd\endcsname{\def\PY@tc##1{\textcolor[rgb]{0.63,0.00,0.00}{##1}}}
\expandafter\def\csname PY@tok@gi\endcsname{\def\PY@tc##1{\textcolor[rgb]{0.00,0.63,0.00}{##1}}}
\expandafter\def\csname PY@tok@gr\endcsname{\def\PY@tc##1{\textcolor[rgb]{1.00,0.00,0.00}{##1}}}
\expandafter\def\csname PY@tok@ge\endcsname{\let\PY@it=\textit}
\expandafter\def\csname PY@tok@gs\endcsname{\let\PY@bf=\textbf}
\expandafter\def\csname PY@tok@gp\endcsname{\let\PY@bf=\textbf\def\PY@tc##1{\textcolor[rgb]{0.00,0.00,0.50}{##1}}}
\expandafter\def\csname PY@tok@go\endcsname{\def\PY@tc##1{\textcolor[rgb]{0.53,0.53,0.53}{##1}}}
\expandafter\def\csname PY@tok@gt\endcsname{\def\PY@tc##1{\textcolor[rgb]{0.00,0.27,0.87}{##1}}}
\expandafter\def\csname PY@tok@err\endcsname{\def\PY@bc##1{\setlength{\fboxsep}{0pt}\fcolorbox[rgb]{1.00,0.00,0.00}{1,1,1}{\strut ##1}}}
\expandafter\def\csname PY@tok@kc\endcsname{\let\PY@bf=\textbf\def\PY@tc##1{\textcolor[rgb]{0.00,0.50,0.00}{##1}}}
\expandafter\def\csname PY@tok@kd\endcsname{\let\PY@bf=\textbf\def\PY@tc##1{\textcolor[rgb]{0.00,0.50,0.00}{##1}}}
\expandafter\def\csname PY@tok@kn\endcsname{\let\PY@bf=\textbf\def\PY@tc##1{\textcolor[rgb]{0.00,0.50,0.00}{##1}}}
\expandafter\def\csname PY@tok@kr\endcsname{\let\PY@bf=\textbf\def\PY@tc##1{\textcolor[rgb]{0.00,0.50,0.00}{##1}}}
\expandafter\def\csname PY@tok@bp\endcsname{\def\PY@tc##1{\textcolor[rgb]{0.00,0.50,0.00}{##1}}}
\expandafter\def\csname PY@tok@fm\endcsname{\def\PY@tc##1{\textcolor[rgb]{0.00,0.00,1.00}{##1}}}
\expandafter\def\csname PY@tok@vc\endcsname{\def\PY@tc##1{\textcolor[rgb]{0.10,0.09,0.49}{##1}}}
\expandafter\def\csname PY@tok@vg\endcsname{\def\PY@tc##1{\textcolor[rgb]{0.10,0.09,0.49}{##1}}}
\expandafter\def\csname PY@tok@vi\endcsname{\def\PY@tc##1{\textcolor[rgb]{0.10,0.09,0.49}{##1}}}
\expandafter\def\csname PY@tok@vm\endcsname{\def\PY@tc##1{\textcolor[rgb]{0.10,0.09,0.49}{##1}}}
\expandafter\def\csname PY@tok@sa\endcsname{\def\PY@tc##1{\textcolor[rgb]{0.73,0.13,0.13}{##1}}}
\expandafter\def\csname PY@tok@sb\endcsname{\def\PY@tc##1{\textcolor[rgb]{0.73,0.13,0.13}{##1}}}
\expandafter\def\csname PY@tok@sc\endcsname{\def\PY@tc##1{\textcolor[rgb]{0.73,0.13,0.13}{##1}}}
\expandafter\def\csname PY@tok@dl\endcsname{\def\PY@tc##1{\textcolor[rgb]{0.73,0.13,0.13}{##1}}}
\expandafter\def\csname PY@tok@s2\endcsname{\def\PY@tc##1{\textcolor[rgb]{0.73,0.13,0.13}{##1}}}
\expandafter\def\csname PY@tok@sh\endcsname{\def\PY@tc##1{\textcolor[rgb]{0.73,0.13,0.13}{##1}}}
\expandafter\def\csname PY@tok@s1\endcsname{\def\PY@tc##1{\textcolor[rgb]{0.73,0.13,0.13}{##1}}}
\expandafter\def\csname PY@tok@mb\endcsname{\def\PY@tc##1{\textcolor[rgb]{0.40,0.40,0.40}{##1}}}
\expandafter\def\csname PY@tok@mf\endcsname{\def\PY@tc##1{\textcolor[rgb]{0.40,0.40,0.40}{##1}}}
\expandafter\def\csname PY@tok@mh\endcsname{\def\PY@tc##1{\textcolor[rgb]{0.40,0.40,0.40}{##1}}}
\expandafter\def\csname PY@tok@mi\endcsname{\def\PY@tc##1{\textcolor[rgb]{0.40,0.40,0.40}{##1}}}
\expandafter\def\csname PY@tok@il\endcsname{\def\PY@tc##1{\textcolor[rgb]{0.40,0.40,0.40}{##1}}}
\expandafter\def\csname PY@tok@mo\endcsname{\def\PY@tc##1{\textcolor[rgb]{0.40,0.40,0.40}{##1}}}
\expandafter\def\csname PY@tok@ch\endcsname{\let\PY@it=\textit\def\PY@tc##1{\textcolor[rgb]{0.25,0.50,0.50}{##1}}}
\expandafter\def\csname PY@tok@cm\endcsname{\let\PY@it=\textit\def\PY@tc##1{\textcolor[rgb]{0.25,0.50,0.50}{##1}}}
\expandafter\def\csname PY@tok@cpf\endcsname{\let\PY@it=\textit\def\PY@tc##1{\textcolor[rgb]{0.25,0.50,0.50}{##1}}}
\expandafter\def\csname PY@tok@c1\endcsname{\let\PY@it=\textit\def\PY@tc##1{\textcolor[rgb]{0.25,0.50,0.50}{##1}}}
\expandafter\def\csname PY@tok@cs\endcsname{\let\PY@it=\textit\def\PY@tc##1{\textcolor[rgb]{0.25,0.50,0.50}{##1}}}

\def\PYZbs{\char`\\}
\def\PYZus{\char`\_}
\def\PYZob{\char`\{}
\def\PYZcb{\char`\}}
\def\PYZca{\char`\^}
\def\PYZam{\char`\&}
\def\PYZlt{\char`\<}
\def\PYZgt{\char`\>}
\def\PYZsh{\char`\#}
\def\PYZpc{\char`\%}
\def\PYZdl{\char`\$}
\def\PYZhy{\char`\-}
\def\PYZsq{\char`\'}
\def\PYZdq{\char`\"}
\def\PYZti{\char`\~}
% for compatibility with earlier versions
\def\PYZat{@}
\def\PYZlb{[}
\def\PYZrb{]}
\makeatother


    % Exact colors from NB
    \definecolor{incolor}{rgb}{0.0, 0.0, 0.5}
    \definecolor{outcolor}{rgb}{0.545, 0.0, 0.0}



    
    % Prevent overflowing lines due to hard-to-break entities
    \sloppy 
    % Setup hyperref package
    \hypersetup{
      breaklinks=true,  % so long urls are correctly broken across lines
      colorlinks=true,
      urlcolor=urlcolor,
      linkcolor=linkcolor,
      citecolor=citecolor,
      }
    % Slightly bigger margins than the latex defaults
    
    \geometry{verbose,tmargin=1in,bmargin=1in,lmargin=1in,rmargin=1in}
    
    

    \begin{document}
    
    
    \maketitle
    
    

    
    \hypertarget{ai-for-medicine-course-1-week-1-lecture-exercises}{%
\subsection{AI for Medicine Course 1 Week 1 lecture
exercises}\label{ai-for-medicine-course-1-week-1-lecture-exercises}}

    \hypertarget{data-exploration}{%
\section{Data Exploration}\label{data-exploration}}

In the first assignment of this course, you will work with chest x-ray
images taken from the public
\href{https://arxiv.org/abs/1705.02315}{ChestX-ray8 dataset}. In this
notebook, you'll get a chance to explore this dataset and familiarize
yourself with some of the techniques you'll use in the first graded
assignment.

The first step before jumping into writing code for any machine learning
project is to explore your data. A standard Python package for analyzing
and manipulating data is
\href{https://pandas.pydata.org/docs/\#}{pandas}.

With the next two code cells, you'll import \texttt{pandas} and a
package called \texttt{numpy} for numerical manipulation, then use
\texttt{pandas} to read a csv file into a dataframe and print out the
first few rows of data.

    \begin{Verbatim}[commandchars=\\\{\}]
{\color{incolor}In [{\color{incolor}1}]:} \PY{c+c1}{\PYZsh{} Import necessary packages}
        \PY{k+kn}{import} \PY{n+nn}{pandas} \PY{k}{as} \PY{n+nn}{pd}
        \PY{k+kn}{import} \PY{n+nn}{numpy} \PY{k}{as} \PY{n+nn}{np}
        \PY{k+kn}{import} \PY{n+nn}{matplotlib}\PY{n+nn}{.}\PY{n+nn}{pyplot} \PY{k}{as} \PY{n+nn}{plt}
        \PY{o}{\PYZpc{}}\PY{k}{matplotlib} inline
        \PY{k+kn}{import} \PY{n+nn}{os}
        \PY{k+kn}{import} \PY{n+nn}{seaborn} \PY{k}{as} \PY{n+nn}{sns}
        \PY{n}{sns}\PY{o}{.}\PY{n}{set}\PY{p}{(}\PY{p}{)}
\end{Verbatim}


    \begin{Verbatim}[commandchars=\\\{\}]
{\color{incolor}In [{\color{incolor}2}]:} \PY{c+c1}{\PYZsh{} Read csv file containing training datadata}
        \PY{n}{train\PYZus{}df} \PY{o}{=} \PY{n}{pd}\PY{o}{.}\PY{n}{read\PYZus{}csv}\PY{p}{(}\PY{l+s+s2}{\PYZdq{}}\PY{l+s+s2}{nih/train\PYZhy{}small.csv}\PY{l+s+s2}{\PYZdq{}}\PY{p}{)}
        \PY{c+c1}{\PYZsh{} Print first 5 rows}
        \PY{n+nb}{print}\PY{p}{(}\PY{n}{f}\PY{l+s+s1}{\PYZsq{}}\PY{l+s+s1}{There are }\PY{l+s+si}{\PYZob{}train\PYZus{}df.shape[0]\PYZcb{}}\PY{l+s+s1}{ rows and }\PY{l+s+si}{\PYZob{}train\PYZus{}df.shape[1]\PYZcb{}}\PY{l+s+s1}{ columns in this data frame}\PY{l+s+s1}{\PYZsq{}}\PY{p}{)}
        \PY{n}{train\PYZus{}df}\PY{o}{.}\PY{n}{head}\PY{p}{(}\PY{p}{)}
\end{Verbatim}


    \begin{Verbatim}[commandchars=\\\{\}]
There are 1000 rows and 16 columns in this data frame

    \end{Verbatim}

\begin{Verbatim}[commandchars=\\\{\}]
{\color{outcolor}Out[{\color{outcolor}2}]:}               Image  Atelectasis  Cardiomegaly  Consolidation  Edema  \textbackslash{}
        0  00008270\_015.png            0             0              0      0   
        1  00029855\_001.png            1             0              0      0   
        2  00001297\_000.png            0             0              0      0   
        3  00012359\_002.png            0             0              0      0   
        4  00017951\_001.png            0             0              0      0   
        
           Effusion  Emphysema  Fibrosis  Hernia  Infiltration  Mass  Nodule  \textbackslash{}
        0         0          0         0       0             0     0       0   
        1         1          0         0       0             1     0       0   
        2         0          0         0       0             0     0       0   
        3         0          0         0       0             0     0       0   
        4         0          0         0       0             1     0       0   
        
           PatientId  Pleural\_Thickening  Pneumonia  Pneumothorax  
        0       8270                   0          0             0  
        1      29855                   0          0             0  
        2       1297                   1          0             0  
        3      12359                   0          0             0  
        4      17951                   0          0             0  
\end{Verbatim}
            
    Have a look at the various columns in this csv file. The file contains
the names of chest x-ray images (``Image'' column) and the columns
filled with ones and zeros identify which diagnoses were given based on
each x-ray image.

    \hypertarget{data-types-and-null-values-check}{%
\subsubsection{Data types and null values
check}\label{data-types-and-null-values-check}}

Run the next cell to explore the data types present in each column and
whether any null values exist in the data.

    \begin{Verbatim}[commandchars=\\\{\}]
{\color{incolor}In [{\color{incolor}3}]:} \PY{c+c1}{\PYZsh{} Look at the data type of each column and whether null values are present}
        \PY{n}{train\PYZus{}df}\PY{o}{.}\PY{n}{info}\PY{p}{(}\PY{p}{)}
\end{Verbatim}


    \begin{Verbatim}[commandchars=\\\{\}]
<class 'pandas.core.frame.DataFrame'>
RangeIndex: 1000 entries, 0 to 999
Data columns (total 16 columns):
Image                 1000 non-null object
Atelectasis           1000 non-null int64
Cardiomegaly          1000 non-null int64
Consolidation         1000 non-null int64
Edema                 1000 non-null int64
Effusion              1000 non-null int64
Emphysema             1000 non-null int64
Fibrosis              1000 non-null int64
Hernia                1000 non-null int64
Infiltration          1000 non-null int64
Mass                  1000 non-null int64
Nodule                1000 non-null int64
PatientId             1000 non-null int64
Pleural\_Thickening    1000 non-null int64
Pneumonia             1000 non-null int64
Pneumothorax          1000 non-null int64
dtypes: int64(15), object(1)
memory usage: 125.1+ KB

    \end{Verbatim}

    \hypertarget{unique-ids-check}{%
\subsubsection{Unique IDs check}\label{unique-ids-check}}

``PatientId'' has an identification number for each patient. One thing
you'd like to know about a medical dataset like this is if you're
looking at repeated data for certain patients or whether each image
represents a different person.

    \begin{Verbatim}[commandchars=\\\{\}]
{\color{incolor}In [{\color{incolor}4}]:} \PY{n+nb}{print}\PY{p}{(}\PY{n}{f}\PY{l+s+s2}{\PYZdq{}}\PY{l+s+s2}{The total patient ids are }\PY{l+s+s2}{\PYZob{}}\PY{l+s+s2}{train\PYZus{}df[}\PY{l+s+s2}{\PYZsq{}}\PY{l+s+s2}{PatientId}\PY{l+s+s2}{\PYZsq{}}\PY{l+s+s2}{].count()\PYZcb{}, from those the unique ids are }\PY{l+s+s2}{\PYZob{}}\PY{l+s+s2}{train\PYZus{}df[}\PY{l+s+s2}{\PYZsq{}}\PY{l+s+s2}{PatientId}\PY{l+s+s2}{\PYZsq{}}\PY{l+s+s2}{].value\PYZus{}counts().shape[0]\PYZcb{} }\PY{l+s+s2}{\PYZdq{}}\PY{p}{)}
\end{Verbatim}


    \begin{Verbatim}[commandchars=\\\{\}]
The total patient ids are 1000, from those the unique ids are 928 

    \end{Verbatim}

    As you can see, the number of unique patients in the dataset is less
than the total number so there must be some overlap. For patients with
multiple records, you'll want to make sure they do not show up in both
training and test sets in order to avoid data leakage (covered later in
this week's lectures).

\hypertarget{explore-data-labels}{%
\subsubsection{Explore data labels}\label{explore-data-labels}}

Run the next two code cells to create a list of the names of each
patient condition or disease.

    \begin{Verbatim}[commandchars=\\\{\}]
{\color{incolor}In [{\color{incolor}5}]:} \PY{n}{columns} \PY{o}{=} \PY{n}{train\PYZus{}df}\PY{o}{.}\PY{n}{keys}\PY{p}{(}\PY{p}{)}
        \PY{n}{columns} \PY{o}{=} \PY{n+nb}{list}\PY{p}{(}\PY{n}{columns}\PY{p}{)}
        \PY{n+nb}{print}\PY{p}{(}\PY{n}{columns}\PY{p}{)}
\end{Verbatim}


    \begin{Verbatim}[commandchars=\\\{\}]
['Image', 'Atelectasis', 'Cardiomegaly', 'Consolidation', 'Edema', 'Effusion', 'Emphysema', 'Fibrosis', 'Hernia', 'Infiltration', 'Mass', 'Nodule', 'PatientId', 'Pleural\_Thickening', 'Pneumonia', 'Pneumothorax']

    \end{Verbatim}

    \begin{Verbatim}[commandchars=\\\{\}]
{\color{incolor}In [{\color{incolor}6}]:} \PY{c+c1}{\PYZsh{} Remove unnecesary elements}
        \PY{n}{columns}\PY{o}{.}\PY{n}{remove}\PY{p}{(}\PY{l+s+s1}{\PYZsq{}}\PY{l+s+s1}{Image}\PY{l+s+s1}{\PYZsq{}}\PY{p}{)}
        \PY{n}{columns}\PY{o}{.}\PY{n}{remove}\PY{p}{(}\PY{l+s+s1}{\PYZsq{}}\PY{l+s+s1}{PatientId}\PY{l+s+s1}{\PYZsq{}}\PY{p}{)}
        \PY{c+c1}{\PYZsh{} Get the total classes}
        \PY{n+nb}{print}\PY{p}{(}\PY{n}{f}\PY{l+s+s2}{\PYZdq{}}\PY{l+s+s2}{There are }\PY{l+s+s2}{\PYZob{}}\PY{l+s+s2}{len(columns)\PYZcb{} columns of labels for these conditions: }\PY{l+s+si}{\PYZob{}columns\PYZcb{}}\PY{l+s+s2}{\PYZdq{}}\PY{p}{)}
\end{Verbatim}


    \begin{Verbatim}[commandchars=\\\{\}]
There are 14 columns of labels for these conditions: ['Atelectasis', 'Cardiomegaly', 'Consolidation', 'Edema', 'Effusion', 'Emphysema', 'Fibrosis', 'Hernia', 'Infiltration', 'Mass', 'Nodule', 'Pleural\_Thickening', 'Pneumonia', 'Pneumothorax']

    \end{Verbatim}

    Run the next cell to print out the number of positive labels (1's) for
each condition

    \begin{Verbatim}[commandchars=\\\{\}]
{\color{incolor}In [{\color{incolor}7}]:} \PY{c+c1}{\PYZsh{} Print out the number of positive labels for each class}
        \PY{k}{for} \PY{n}{column} \PY{o+ow}{in} \PY{n}{columns}\PY{p}{:}
            \PY{n+nb}{print}\PY{p}{(}\PY{n}{f}\PY{l+s+s2}{\PYZdq{}}\PY{l+s+s2}{The class }\PY{l+s+si}{\PYZob{}column\PYZcb{}}\PY{l+s+s2}{ has }\PY{l+s+s2}{\PYZob{}}\PY{l+s+s2}{train\PYZus{}df[column].sum()\PYZcb{} samples}\PY{l+s+s2}{\PYZdq{}}\PY{p}{)}
\end{Verbatim}


    \begin{Verbatim}[commandchars=\\\{\}]
The class Atelectasis has 106 samples
The class Cardiomegaly has 20 samples
The class Consolidation has 33 samples
The class Edema has 16 samples
The class Effusion has 128 samples
The class Emphysema has 13 samples
The class Fibrosis has 14 samples
The class Hernia has 2 samples
The class Infiltration has 175 samples
The class Mass has 45 samples
The class Nodule has 54 samples
The class Pleural\_Thickening has 21 samples
The class Pneumonia has 10 samples
The class Pneumothorax has 38 samples

    \end{Verbatim}

    Have a look at the counts for the labels in each class above. Does this
look like a balanced dataset?

    \hypertarget{data-visualization}{%
\subsubsection{Data Visualization}\label{data-visualization}}

Using the image names listed in the csv file, you can retrieve the image
associated with each row of data in your dataframe.

Run the cell below to visualize a random selection of images from the
dataset.

    \begin{Verbatim}[commandchars=\\\{\}]
{\color{incolor}In [{\color{incolor}8}]:} \PY{c+c1}{\PYZsh{} Extract numpy values from Image column in data frame}
        \PY{n}{images} \PY{o}{=} \PY{n}{train\PYZus{}df}\PY{p}{[}\PY{l+s+s1}{\PYZsq{}}\PY{l+s+s1}{Image}\PY{l+s+s1}{\PYZsq{}}\PY{p}{]}\PY{o}{.}\PY{n}{values}
        
        \PY{c+c1}{\PYZsh{} Extract 9 random images from it}
        \PY{n}{random\PYZus{}images} \PY{o}{=} \PY{p}{[}\PY{n}{np}\PY{o}{.}\PY{n}{random}\PY{o}{.}\PY{n}{choice}\PY{p}{(}\PY{n}{images}\PY{p}{)} \PY{k}{for} \PY{n}{i} \PY{o+ow}{in} \PY{n+nb}{range}\PY{p}{(}\PY{l+m+mi}{9}\PY{p}{)}\PY{p}{]}
        
        \PY{c+c1}{\PYZsh{} Location of the image dir}
        \PY{n}{img\PYZus{}dir} \PY{o}{=} \PY{l+s+s1}{\PYZsq{}}\PY{l+s+s1}{nih/images\PYZhy{}small/}\PY{l+s+s1}{\PYZsq{}}
        
        \PY{n+nb}{print}\PY{p}{(}\PY{l+s+s1}{\PYZsq{}}\PY{l+s+s1}{Display Random Images}\PY{l+s+s1}{\PYZsq{}}\PY{p}{)}
        
        \PY{c+c1}{\PYZsh{} Adjust the size of your images}
        \PY{n}{plt}\PY{o}{.}\PY{n}{figure}\PY{p}{(}\PY{n}{figsize}\PY{o}{=}\PY{p}{(}\PY{l+m+mi}{20}\PY{p}{,}\PY{l+m+mi}{10}\PY{p}{)}\PY{p}{)}
        
        \PY{c+c1}{\PYZsh{} Iterate and plot random images}
        \PY{k}{for} \PY{n}{i} \PY{o+ow}{in} \PY{n+nb}{range}\PY{p}{(}\PY{l+m+mi}{9}\PY{p}{)}\PY{p}{:}
            \PY{n}{plt}\PY{o}{.}\PY{n}{subplot}\PY{p}{(}\PY{l+m+mi}{3}\PY{p}{,} \PY{l+m+mi}{3}\PY{p}{,} \PY{n}{i} \PY{o}{+} \PY{l+m+mi}{1}\PY{p}{)}
            \PY{n}{img} \PY{o}{=} \PY{n}{plt}\PY{o}{.}\PY{n}{imread}\PY{p}{(}\PY{n}{os}\PY{o}{.}\PY{n}{path}\PY{o}{.}\PY{n}{join}\PY{p}{(}\PY{n}{img\PYZus{}dir}\PY{p}{,} \PY{n}{random\PYZus{}images}\PY{p}{[}\PY{n}{i}\PY{p}{]}\PY{p}{)}\PY{p}{)}
            \PY{n}{plt}\PY{o}{.}\PY{n}{imshow}\PY{p}{(}\PY{n}{img}\PY{p}{,} \PY{n}{cmap}\PY{o}{=}\PY{l+s+s1}{\PYZsq{}}\PY{l+s+s1}{gray}\PY{l+s+s1}{\PYZsq{}}\PY{p}{)}
            \PY{n}{plt}\PY{o}{.}\PY{n}{axis}\PY{p}{(}\PY{l+s+s1}{\PYZsq{}}\PY{l+s+s1}{off}\PY{l+s+s1}{\PYZsq{}}\PY{p}{)}
            
        \PY{c+c1}{\PYZsh{} Adjust subplot parameters to give specified padding}
        \PY{n}{plt}\PY{o}{.}\PY{n}{tight\PYZus{}layout}\PY{p}{(}\PY{p}{)}    
\end{Verbatim}


    \begin{Verbatim}[commandchars=\\\{\}]
Display Random Images

    \end{Verbatim}

    \begin{center}
    \adjustimage{max size={0.9\linewidth}{0.9\paperheight}}{output_16_1.png}
    \end{center}
    { \hspace*{\fill} \\}
    
    \hypertarget{investigate-a-single-image}{%
\subsubsection{Investigate a single
image}\label{investigate-a-single-image}}

Run the cell below to look at the first image in the dataset and print
out some details of the image contents.

    \begin{Verbatim}[commandchars=\\\{\}]
{\color{incolor}In [{\color{incolor}9}]:} \PY{c+c1}{\PYZsh{} Get the first image that was listed in the train\PYZus{}df dataframe}
        \PY{n}{sample\PYZus{}img} \PY{o}{=} \PY{n}{train\PYZus{}df}\PY{o}{.}\PY{n}{Image}\PY{p}{[}\PY{l+m+mi}{0}\PY{p}{]}
        \PY{n}{raw\PYZus{}image} \PY{o}{=} \PY{n}{plt}\PY{o}{.}\PY{n}{imread}\PY{p}{(}\PY{n}{os}\PY{o}{.}\PY{n}{path}\PY{o}{.}\PY{n}{join}\PY{p}{(}\PY{n}{img\PYZus{}dir}\PY{p}{,} \PY{n}{sample\PYZus{}img}\PY{p}{)}\PY{p}{)}
        \PY{n}{plt}\PY{o}{.}\PY{n}{imshow}\PY{p}{(}\PY{n}{raw\PYZus{}image}\PY{p}{,} \PY{n}{cmap}\PY{o}{=}\PY{l+s+s1}{\PYZsq{}}\PY{l+s+s1}{gray}\PY{l+s+s1}{\PYZsq{}}\PY{p}{)}
        \PY{n}{plt}\PY{o}{.}\PY{n}{colorbar}\PY{p}{(}\PY{p}{)}
        \PY{n}{plt}\PY{o}{.}\PY{n}{title}\PY{p}{(}\PY{l+s+s1}{\PYZsq{}}\PY{l+s+s1}{Raw Chest X Ray Image}\PY{l+s+s1}{\PYZsq{}}\PY{p}{)}
        \PY{n+nb}{print}\PY{p}{(}\PY{n}{f}\PY{l+s+s2}{\PYZdq{}}\PY{l+s+s2}{The dimensions of the image are }\PY{l+s+si}{\PYZob{}raw\PYZus{}image.shape[0]\PYZcb{}}\PY{l+s+s2}{ pixels width and }\PY{l+s+si}{\PYZob{}raw\PYZus{}image.shape[1]\PYZcb{}}\PY{l+s+s2}{ pixels height, one single color channel}\PY{l+s+s2}{\PYZdq{}}\PY{p}{)}
        \PY{n+nb}{print}\PY{p}{(}\PY{n}{f}\PY{l+s+s2}{\PYZdq{}}\PY{l+s+s2}{The maximum pixel value is }\PY{l+s+s2}{\PYZob{}}\PY{l+s+s2}{raw\PYZus{}image.max():.4f\PYZcb{} and the minimum is }\PY{l+s+s2}{\PYZob{}}\PY{l+s+s2}{raw\PYZus{}image.min():.4f\PYZcb{}}\PY{l+s+s2}{\PYZdq{}}\PY{p}{)}
        \PY{n+nb}{print}\PY{p}{(}\PY{n}{f}\PY{l+s+s2}{\PYZdq{}}\PY{l+s+s2}{The mean value of the pixels is }\PY{l+s+s2}{\PYZob{}}\PY{l+s+s2}{raw\PYZus{}image.mean():.4f\PYZcb{} and the standard deviation is }\PY{l+s+s2}{\PYZob{}}\PY{l+s+s2}{raw\PYZus{}image.std():.4f\PYZcb{}}\PY{l+s+s2}{\PYZdq{}}\PY{p}{)}
\end{Verbatim}


    \begin{Verbatim}[commandchars=\\\{\}]
The dimensions of the image are 1024 pixels width and 1024 pixels height, one single color channel
The maximum pixel value is 0.9804 and the minimum is 0.0000
The mean value of the pixels is 0.4796 and the standard deviation is 0.2757

    \end{Verbatim}

    \begin{center}
    \adjustimage{max size={0.9\linewidth}{0.9\paperheight}}{output_18_1.png}
    \end{center}
    { \hspace*{\fill} \\}
    
    \hypertarget{investigate-pixel-value-distribution}{%
\subsubsection{Investigate pixel value
distribution}\label{investigate-pixel-value-distribution}}

Run the cell below to plot up the distribution of pixel values in the
image shown above.

    \begin{Verbatim}[commandchars=\\\{\}]
{\color{incolor}In [{\color{incolor}10}]:} \PY{c+c1}{\PYZsh{} Plot a histogram of the distribution of the pixels}
         \PY{n}{sns}\PY{o}{.}\PY{n}{distplot}\PY{p}{(}\PY{n}{raw\PYZus{}image}\PY{o}{.}\PY{n}{ravel}\PY{p}{(}\PY{p}{)}\PY{p}{,} 
                      \PY{n}{label}\PY{o}{=}\PY{n}{f}\PY{l+s+s1}{\PYZsq{}}\PY{l+s+s1}{Pixel Mean }\PY{l+s+s1}{\PYZob{}}\PY{l+s+s1}{np.mean(raw\PYZus{}image):.4f\PYZcb{} \PYZam{} Standard Deviation }\PY{l+s+s1}{\PYZob{}}\PY{l+s+s1}{np.std(raw\PYZus{}image):.4f\PYZcb{}}\PY{l+s+s1}{\PYZsq{}}\PY{p}{,} \PY{n}{kde}\PY{o}{=}\PY{k+kc}{False}\PY{p}{)}
         \PY{n}{plt}\PY{o}{.}\PY{n}{legend}\PY{p}{(}\PY{n}{loc}\PY{o}{=}\PY{l+s+s1}{\PYZsq{}}\PY{l+s+s1}{upper center}\PY{l+s+s1}{\PYZsq{}}\PY{p}{)}
         \PY{n}{plt}\PY{o}{.}\PY{n}{title}\PY{p}{(}\PY{l+s+s1}{\PYZsq{}}\PY{l+s+s1}{Distribution of Pixel Intensities in the Image}\PY{l+s+s1}{\PYZsq{}}\PY{p}{)}
         \PY{n}{plt}\PY{o}{.}\PY{n}{xlabel}\PY{p}{(}\PY{l+s+s1}{\PYZsq{}}\PY{l+s+s1}{Pixel Intensity}\PY{l+s+s1}{\PYZsq{}}\PY{p}{)}
         \PY{n}{plt}\PY{o}{.}\PY{n}{ylabel}\PY{p}{(}\PY{l+s+s1}{\PYZsq{}}\PY{l+s+s1}{\PYZsh{} Pixels in Image}\PY{l+s+s1}{\PYZsq{}}\PY{p}{)}
\end{Verbatim}


\begin{Verbatim}[commandchars=\\\{\}]
{\color{outcolor}Out[{\color{outcolor}10}]:} Text(0, 0.5, '\# Pixels in Image')
\end{Verbatim}
            
    \begin{center}
    \adjustimage{max size={0.9\linewidth}{0.9\paperheight}}{output_20_1.png}
    \end{center}
    { \hspace*{\fill} \\}
    
    \hypertarget{image-preprocessing-in-keras}{%
\section{Image Preprocessing in
Keras}\label{image-preprocessing-in-keras}}

Before training, you'll first modify your images to be better suited for
training a convolutional neural network. For this task you'll use the
Keras \href{https://keras.io/preprocessing/image/}{ImageDataGenerator}
function to perform data preprocessing and data augmentation.

Run the next two cells to import this function and create an image
generator for preprocessing.

    \begin{Verbatim}[commandchars=\\\{\}]
{\color{incolor}In [{\color{incolor}11}]:} \PY{c+c1}{\PYZsh{} Import data generator from keras}
         \PY{k+kn}{from} \PY{n+nn}{keras}\PY{n+nn}{.}\PY{n+nn}{preprocessing}\PY{n+nn}{.}\PY{n+nn}{image} \PY{k}{import} \PY{n}{ImageDataGenerator}
\end{Verbatim}


    \begin{Verbatim}[commandchars=\\\{\}]
Using TensorFlow backend.

    \end{Verbatim}

    \begin{Verbatim}[commandchars=\\\{\}]
{\color{incolor}In [{\color{incolor}12}]:} \PY{c+c1}{\PYZsh{} Normalize images}
         \PY{n}{image\PYZus{}generator} \PY{o}{=} \PY{n}{ImageDataGenerator}\PY{p}{(}
             \PY{n}{samplewise\PYZus{}center}\PY{o}{=}\PY{k+kc}{True}\PY{p}{,} \PY{c+c1}{\PYZsh{}Set each sample mean to 0.}
             \PY{n}{samplewise\PYZus{}std\PYZus{}normalization}\PY{o}{=} \PY{k+kc}{True} \PY{c+c1}{\PYZsh{} Divide each input by its standard deviation}
         \PY{p}{)}
\end{Verbatim}


    \hypertarget{standardization}{%
\subsubsection{Standardization}\label{standardization}}

The \texttt{image\_generator} you created above will act to adjust your
image data such that the new mean of the data will be zero, and the
standard deviation of the data will be 1.

In other words, the generator will replace each pixel value in the image
with a new value calculated by subtracting the mean and dividing by the
standard deviation.

\[\frac{x_i - \mu}{\sigma}\]

Run the next cell to pre-process your data using the
\texttt{image\_generator}. In this step you will also be reducing the
image size down to 320x320 pixels.

    \begin{Verbatim}[commandchars=\\\{\}]
{\color{incolor}In [{\color{incolor}13}]:} \PY{c+c1}{\PYZsh{} Flow from directory with specified batch size and target image size}
         \PY{n}{generator} \PY{o}{=} \PY{n}{image\PYZus{}generator}\PY{o}{.}\PY{n}{flow\PYZus{}from\PYZus{}dataframe}\PY{p}{(}
                 \PY{n}{dataframe}\PY{o}{=}\PY{n}{train\PYZus{}df}\PY{p}{,}
                 \PY{n}{directory}\PY{o}{=}\PY{l+s+s2}{\PYZdq{}}\PY{l+s+s2}{nih/images\PYZhy{}small/}\PY{l+s+s2}{\PYZdq{}}\PY{p}{,}
                 \PY{n}{x\PYZus{}col}\PY{o}{=}\PY{l+s+s2}{\PYZdq{}}\PY{l+s+s2}{Image}\PY{l+s+s2}{\PYZdq{}}\PY{p}{,} \PY{c+c1}{\PYZsh{} features}
                 \PY{c+c1}{\PYZsh{} Let\PYZsq{}s say we build a model for mass detection}
                 \PY{n}{y\PYZus{}col}\PY{o}{=} \PY{p}{[}\PY{l+s+s1}{\PYZsq{}}\PY{l+s+s1}{Mass}\PY{l+s+s1}{\PYZsq{}}\PY{p}{]}\PY{p}{,} \PY{c+c1}{\PYZsh{} labels}
                 \PY{n}{class\PYZus{}mode}\PY{o}{=}\PY{l+s+s2}{\PYZdq{}}\PY{l+s+s2}{raw}\PY{l+s+s2}{\PYZdq{}}\PY{p}{,} \PY{c+c1}{\PYZsh{} \PYZsq{}Mass\PYZsq{} column should be in train\PYZus{}df}
                 \PY{n}{batch\PYZus{}size}\PY{o}{=} \PY{l+m+mi}{1}\PY{p}{,} \PY{c+c1}{\PYZsh{} images per batch}
                 \PY{n}{shuffle}\PY{o}{=}\PY{k+kc}{False}\PY{p}{,} \PY{c+c1}{\PYZsh{} shuffle the rows or not}
                 \PY{n}{target\PYZus{}size}\PY{o}{=}\PY{p}{(}\PY{l+m+mi}{320}\PY{p}{,}\PY{l+m+mi}{320}\PY{p}{)} \PY{c+c1}{\PYZsh{} width and height of output image}
         \PY{p}{)}
\end{Verbatim}


    \begin{Verbatim}[commandchars=\\\{\}]
Found 1000 validated image filenames.

    \end{Verbatim}

    Run the next cell to plot up an example of a pre-processed image

    \begin{Verbatim}[commandchars=\\\{\}]
{\color{incolor}In [{\color{incolor}14}]:} \PY{c+c1}{\PYZsh{} Plot a processed image}
         \PY{n}{sns}\PY{o}{.}\PY{n}{set\PYZus{}style}\PY{p}{(}\PY{l+s+s2}{\PYZdq{}}\PY{l+s+s2}{white}\PY{l+s+s2}{\PYZdq{}}\PY{p}{)}
         \PY{n}{generated\PYZus{}image}\PY{p}{,} \PY{n}{label} \PY{o}{=} \PY{n}{generator}\PY{o}{.}\PY{n+nf+fm}{\PYZus{}\PYZus{}getitem\PYZus{}\PYZus{}}\PY{p}{(}\PY{l+m+mi}{0}\PY{p}{)}
         \PY{n}{plt}\PY{o}{.}\PY{n}{imshow}\PY{p}{(}\PY{n}{generated\PYZus{}image}\PY{p}{[}\PY{l+m+mi}{0}\PY{p}{]}\PY{p}{,} \PY{n}{cmap}\PY{o}{=}\PY{l+s+s1}{\PYZsq{}}\PY{l+s+s1}{gray}\PY{l+s+s1}{\PYZsq{}}\PY{p}{)}
         \PY{n}{plt}\PY{o}{.}\PY{n}{colorbar}\PY{p}{(}\PY{p}{)}
         \PY{n}{plt}\PY{o}{.}\PY{n}{title}\PY{p}{(}\PY{l+s+s1}{\PYZsq{}}\PY{l+s+s1}{Raw Chest X Ray Image}\PY{l+s+s1}{\PYZsq{}}\PY{p}{)}
         \PY{n+nb}{print}\PY{p}{(}\PY{n}{f}\PY{l+s+s2}{\PYZdq{}}\PY{l+s+s2}{The dimensions of the image are }\PY{l+s+si}{\PYZob{}generated\PYZus{}image.shape[1]\PYZcb{}}\PY{l+s+s2}{ pixels width and }\PY{l+s+si}{\PYZob{}generated\PYZus{}image.shape[2]\PYZcb{}}\PY{l+s+s2}{ pixels height}\PY{l+s+s2}{\PYZdq{}}\PY{p}{)}
         \PY{n+nb}{print}\PY{p}{(}\PY{n}{f}\PY{l+s+s2}{\PYZdq{}}\PY{l+s+s2}{The maximum pixel value is }\PY{l+s+s2}{\PYZob{}}\PY{l+s+s2}{generated\PYZus{}image.max():.4f\PYZcb{} and the minimum is }\PY{l+s+s2}{\PYZob{}}\PY{l+s+s2}{generated\PYZus{}image.min():.4f\PYZcb{}}\PY{l+s+s2}{\PYZdq{}}\PY{p}{)}
         \PY{n+nb}{print}\PY{p}{(}\PY{n}{f}\PY{l+s+s2}{\PYZdq{}}\PY{l+s+s2}{The mean value of the pixels is }\PY{l+s+s2}{\PYZob{}}\PY{l+s+s2}{generated\PYZus{}image.mean():.4f\PYZcb{} and the standard deviation is }\PY{l+s+s2}{\PYZob{}}\PY{l+s+s2}{generated\PYZus{}image.std():.4f\PYZcb{}}\PY{l+s+s2}{\PYZdq{}}\PY{p}{)}
\end{Verbatim}


    \begin{Verbatim}[commandchars=\\\{\}]
Clipping input data to the valid range for imshow with RGB data ([0..1] for floats or [0..255] for integers).

    \end{Verbatim}

    \begin{Verbatim}[commandchars=\\\{\}]
The dimensions of the image are 320 pixels width and 320 pixels height
The maximum pixel value is 1.7999 and the minimum is -1.7404
The mean value of the pixels is 0.0000 and the standard deviation is 1.0000

    \end{Verbatim}

    \begin{center}
    \adjustimage{max size={0.9\linewidth}{0.9\paperheight}}{output_27_2.png}
    \end{center}
    { \hspace*{\fill} \\}
    
    Run the cell below to see a comparison of the distribution of pixel
values in the new pre-processed image versus the raw image.

    \begin{Verbatim}[commandchars=\\\{\}]
{\color{incolor}In [{\color{incolor}15}]:} \PY{c+c1}{\PYZsh{} Include a histogram of the distribution of the pixels}
         \PY{n}{sns}\PY{o}{.}\PY{n}{set}\PY{p}{(}\PY{p}{)}
         \PY{n}{plt}\PY{o}{.}\PY{n}{figure}\PY{p}{(}\PY{n}{figsize}\PY{o}{=}\PY{p}{(}\PY{l+m+mi}{10}\PY{p}{,} \PY{l+m+mi}{7}\PY{p}{)}\PY{p}{)}
         
         \PY{c+c1}{\PYZsh{} Plot histogram for original iamge}
         \PY{n}{sns}\PY{o}{.}\PY{n}{distplot}\PY{p}{(}\PY{n}{raw\PYZus{}image}\PY{o}{.}\PY{n}{ravel}\PY{p}{(}\PY{p}{)}\PY{p}{,} 
                      \PY{n}{label}\PY{o}{=}\PY{n}{f}\PY{l+s+s1}{\PYZsq{}}\PY{l+s+s1}{Original Image: mean }\PY{l+s+s1}{\PYZob{}}\PY{l+s+s1}{np.mean(raw\PYZus{}image):.4f\PYZcb{} \PYZhy{} Standard Deviation }\PY{l+s+s1}{\PYZob{}}\PY{l+s+s1}{np.std(raw\PYZus{}image):.4f\PYZcb{} }\PY{l+s+se}{\PYZbs{}n}\PY{l+s+s1}{ }\PY{l+s+s1}{\PYZsq{}}
                      \PY{n}{f}\PY{l+s+s1}{\PYZsq{}}\PY{l+s+s1}{Min pixel value }\PY{l+s+s1}{\PYZob{}}\PY{l+s+s1}{np.min(raw\PYZus{}image):.4\PYZcb{} \PYZhy{} Max pixel value }\PY{l+s+s1}{\PYZob{}}\PY{l+s+s1}{np.max(raw\PYZus{}image):.4\PYZcb{}}\PY{l+s+s1}{\PYZsq{}}\PY{p}{,}
                      \PY{n}{color}\PY{o}{=}\PY{l+s+s1}{\PYZsq{}}\PY{l+s+s1}{blue}\PY{l+s+s1}{\PYZsq{}}\PY{p}{,} 
                      \PY{n}{kde}\PY{o}{=}\PY{k+kc}{False}\PY{p}{)}
         
         \PY{c+c1}{\PYZsh{} Plot histogram for generated image}
         \PY{n}{sns}\PY{o}{.}\PY{n}{distplot}\PY{p}{(}\PY{n}{generated\PYZus{}image}\PY{p}{[}\PY{l+m+mi}{0}\PY{p}{]}\PY{o}{.}\PY{n}{ravel}\PY{p}{(}\PY{p}{)}\PY{p}{,} 
                      \PY{n}{label}\PY{o}{=}\PY{n}{f}\PY{l+s+s1}{\PYZsq{}}\PY{l+s+s1}{Generated Image: mean }\PY{l+s+s1}{\PYZob{}}\PY{l+s+s1}{np.mean(generated\PYZus{}image[0]):.4f\PYZcb{} \PYZhy{} Standard Deviation }\PY{l+s+s1}{\PYZob{}}\PY{l+s+s1}{np.std(generated\PYZus{}image[0]):.4f\PYZcb{} }\PY{l+s+se}{\PYZbs{}n}\PY{l+s+s1}{\PYZsq{}}
                      \PY{n}{f}\PY{l+s+s1}{\PYZsq{}}\PY{l+s+s1}{Min pixel value }\PY{l+s+s1}{\PYZob{}}\PY{l+s+s1}{np.min(generated\PYZus{}image[0]):.4\PYZcb{} \PYZhy{} Max pixel value }\PY{l+s+s1}{\PYZob{}}\PY{l+s+s1}{np.max(generated\PYZus{}image[0]):.4\PYZcb{}}\PY{l+s+s1}{\PYZsq{}}\PY{p}{,} 
                      \PY{n}{color}\PY{o}{=}\PY{l+s+s1}{\PYZsq{}}\PY{l+s+s1}{red}\PY{l+s+s1}{\PYZsq{}}\PY{p}{,} 
                      \PY{n}{kde}\PY{o}{=}\PY{k+kc}{False}\PY{p}{)}
         
         \PY{c+c1}{\PYZsh{} Place legends}
         \PY{n}{plt}\PY{o}{.}\PY{n}{legend}\PY{p}{(}\PY{p}{)}
         \PY{n}{plt}\PY{o}{.}\PY{n}{title}\PY{p}{(}\PY{l+s+s1}{\PYZsq{}}\PY{l+s+s1}{Distribution of Pixel Intensities in the Image}\PY{l+s+s1}{\PYZsq{}}\PY{p}{)}
         \PY{n}{plt}\PY{o}{.}\PY{n}{xlabel}\PY{p}{(}\PY{l+s+s1}{\PYZsq{}}\PY{l+s+s1}{Pixel Intensity}\PY{l+s+s1}{\PYZsq{}}\PY{p}{)}
         \PY{n}{plt}\PY{o}{.}\PY{n}{ylabel}\PY{p}{(}\PY{l+s+s1}{\PYZsq{}}\PY{l+s+s1}{\PYZsh{} Pixel}\PY{l+s+s1}{\PYZsq{}}\PY{p}{)}
\end{Verbatim}


\begin{Verbatim}[commandchars=\\\{\}]
{\color{outcolor}Out[{\color{outcolor}15}]:} Text(0, 0.5, '\# Pixel')
\end{Verbatim}
            
    \begin{center}
    \adjustimage{max size={0.9\linewidth}{0.9\paperheight}}{output_29_1.png}
    \end{center}
    { \hspace*{\fill} \\}
    
    \hypertarget{thats-it-for-this-exercise-you-should-now-be-a-bit-more-familiar-with-the-dataset-youll-be-using-in-this-weeks-assignment}{%
\paragraph{That's it for this exercise, you should now be a bit more
familiar with the dataset you'll be using in this week's
assignment!}\label{thats-it-for-this-exercise-you-should-now-be-a-bit-more-familiar-with-the-dataset-youll-be-using-in-this-weeks-assignment}}


    % Add a bibliography block to the postdoc
    
    
    
    \end{document}
